

	One of my friends, a fan of lesser well known films, recently recommended this strange film from 2012, called \textit{The Perks of Being a Wallflower}. His only attempt to convince me to do so was summed up in the phrase "It's really good and has Emma Watson in it". He had my attention. But it was a vague memory of that title from somewhere that held my interest. A little initial research about this film revealed that it had been a book first, published in 1999, written by the author Stephen Chbosky. My interest was piqued; film adaptations tend to be based on good books (with the notable exception of \textit{Twilight}). So I found a copy from the library and started reading...

	One is immediately struck by the way it was written. Instead of being a narrative, the book is presented as a series of letters, to a person who is never identified, stating that the person who he is writing to seems kind and wouldn't be judgmental. However, I felt, reading these personal letters, that they were addressed to the reader. I was also struck by the sheer sense of nostalgia for teenage years the author was able to evoke.

\subsection{Plot}

	The book starts off with our main character, Charlie, a high school freshman, explaining why he is writing these letters believing that the reader will not judge him. He begins by talking about his first year at high school, while still in the grip of two traumatic experiences in his past: the death of his favorite aunt, Helen, in a car crash when he was young, and the tragic suicide of his middle school friend Michael, the year before. He immediately takes a liking for English and his teacher, who insists that Charlie call him Bill, takes notice of his proficiency for the subject. Bill mentors him by setting him extra work and gives him more books to read, which he enjoys greatly, including \textit{The Catcher in the Rye}. Despite being highly introverted, he is befriended by two seniors, Sam and Patrick, Charlie quickly falls for Sam, and admits his feelings for her, Sam rejects his offer kindly, saying that he is just too young for her, but they remain good friends. Patrick, Sam's stepbrother, is found out to be secretly dating Brad, the closeted football quarterback, who is very insistent on keeping their relationship secret.

	Charlie becomes more and more accepted in Sam and Patrick's friend group, and begins to experiment with drugs and alcohol. The high point of their friendship is when all three of them ride in the pickup truck through a tunnel, and Charlie states that he felt infinite. It is revealed that Sam was sexually abused as a child, and kisses Charlie one night to ensure that his first kiss is with someone he likes. At a party on New Year's Eve, Charlie trips on LSD and cannot control flashbacks of his aunt Helen, and ends up in the hospital after passing out in the snow. Charlie is asked to fill in for the main part in a regular performance of \textit{The Rocky Horror Picture Show} instead of Sam's boyfriend, Craig, This impresses one of their friends, Mary Elizabeth, who then asks him to the school's Sadie Hawkins dance. The two then enter a half hearted relationship, which ends quickly, after a game of truth and dare. Charlie is dared to kiss the prettiest girl in the room, but instead of kissing Mary Elizabeth, he kisses Sam, saying to the reader that he needed to be honest. Patrick suggests that he spend some time away from the group, during which, his flashbacks of Aunt Helen return.

	Also during this time, Patrick and Brad's relationship is discovered by Brad's abusive father, who pulls him out of school for a while. On returning, Brad is cold and derogatory to Patrick, even publicly mocking his sexuality in the middle of the dining hall. Patrick snaps, attacking Brad, but is then outnumbered by five other football team members who join him. Charlie rushes to his aid and breaks up the fight, which improves the relationship between him and Sam, as he regains her respect. Patrick starts spending more time with Charlie, and engages in unhealthy behavior, including going to a park to "fool around" with strangers, which Charlie does not try to stop. Eventually Patrick sees Brad engaging with a stranger in the same park and is able to move on from their relationship.

	The school year winds to a close and Charlie is afraid to lose all of his senior friends, especially Sam, who is leaving earlier than the rest, and has just broken up with her boyfriend. In the emotional climax of the book, the two talk about his feelings for her. Sam becomes angry at Charlie for never acting on his feelings for her, saying that she does not want to be someone's crush, because that means they aren't being themselves around her, and are hiding something. She then encourages him to act on the things he feels in all aspects of life, telling him to "participate in life", not just watch from the sidelines as a wallflower. The two begin to engage sexually, but Charlie is suddenly, inexplicably uncomfortable, and realizes that the contact is bringing back repressed memories of him being sexually abused as a child by his aunt Helen.

	In the epilogue, Charlie is found in a catatonic state and is admitted into a mental hospital to help deal with his PTSD. Patrick and Sam visit, which helps his state and he soon comes to terms with his troubled past ``Even if we don't have the power to choose where we come from, we can still choose where we go from there'', and accepts that he needs to participate in life. The book ends with Sam and Patrick driving through the tunnel one last time, and Charlie states again that he feels infinite.

\subsection{Analysis}

	The central theme of the book can be found in the title, as it suggests, \textit{The Perks of Being a Wallflower} is all about Charlie wrestling between observing life, as a wallflower, and internalizing his emotions or to taking part in it and sharing how he feels with others. At the start of the school year, Charlie, we are told, feels like an outcast, knowing hardly anyone and having lost one of his best friends, Michael. But during this troubled times, two important events happen to him, which enrich his life and help him feel special in a good way. Firstly his English teacher takes note of him and sets him extra assignments and reading, all the books he sets are coming of age stories, like \textit{The Perks of Being a Wallflower}, giving a clever nod to its predecessors. This allows him to feel special intellectually. He also meets Sam and Patrick, who take him under their wing, making him feel special socially. Charlie ends up being surprisingly direct in his letters, although he is not at all in his own life, and uses them to externalise his emotions, rather than keep them to himself. For example, when he witnesses his sister being hit by her boyfriend, he writes the letter first, then a month later tells Bill, showing how he uses the letters to verbalise how he feels, then acts on it. The entire novel deals with this issue and Charlie's growth revolves around how he has to come to terms with the things he keeps hidden inside of him. When he relates the fact that he witnessed a date rape to Patrick and Sam, years after it happened, they encourage him to puncture the date rapist's tires, allowing him to feel rewarded and provides emotional release. However he soon learns that some things are best kept to oneself, and uses the letters to work through these, rather than directly acting on them. For example, when he kisses Sam instead of Mary Elizabeth at the party.

	The book ends in an optimistic tone, when, in the epilogue, all three of them ride through the tunnel again. This acts as a powerful expression of nostalgia for Charlie, as it reminds him of the things the friends used to do before Sam and Patrick left. Then the city comes into view, symbolizing how Charlie is ready to start participating in life, completing his growth. The letters stop, showing how he is finally at peace with his  past and who he is.

\subsection{Conclusion}

	I would thoroughly recommend this book as it's a really well developed coming of age story, with a wide variety of themes including: passion vs passivity, sex, drugs, loss and a real sense of nostalgia for the highs and lows of teenage life. Seeing the world through Charlie's eyes gives the reader a good perspective on what it means to take part in life as well as growing up. In the words of Chbosky himself, this book is "for anyone who has ever felt like an outcast", which, given its large cult following, must be a lot of people.
