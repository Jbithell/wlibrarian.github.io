These longer editions are now to be referred to as ``Supplementary Issues'', as many pieces are only tangentially related to the library. The one page editions of \textit{Library News} will now be renamed \textit{The Librarian}.

This issue's dialectic piece, written by James Bithell, explores the nature of the prosecution of politicians. ``Legislating to `protect the public from themselves' is a dangerous game'', as is prosecution for ```lying' or for `aggression{'}'', he writes, albeit conscious that he has ``raised just as many questions'' as he has ``attempted to answer''.

Isky Mathews in the fourth installation of Adventures in Recreational Mathematics confesses to his addiction to sequences, and invites others to join him. The cover photo is from http://swarminglogic.com. \footnote{http://swarminglogic.com/article/2013__oeis} and is a representation of how ``interesting'' a number is based on the number of entries it has in the Online Encyclopedia of Integer Sequences. The website entry notes that the concept of ``interest'' seems somewhat subjective. In this case, it refers to the number of times that a number occurs in a sequence; if this number is large, this suggests that a number is involved in many phenomena, not just a few. It is interesting to note that despite what one might initially suspect, the cover photo shows a number of interesting patterns - the two rectangles and lines are somewhat surprising. 

Benedict Randall Shaw in ``An introduction to Areal coördinates'' introduces readers to the concept of Areal coördinates. Readers are guided through the concept with a sample question.

\textit{The Librarian}'s staff have written summaries and commentary on a number of lectures and meetings. This edition includes:
\begin{itemize}
	\item Emeritus Professor Michael Bentley on \textit{The Gladstone Problem},
	\item The first Imperial College Lecture, viz. Professor Alyssa Aspel on \textit{Wireless Circuits in the Age of the Internet of Things},
	\item Dr. Rufus Duits on \textit{What's the wrong thing to do?}, and
	\item The first meeting of Stipatores Honorati.
\end{itemize}

Thomas Adamo's review of \textit{The perks of being a wallflower} is the penultimate article. Adamo writes that he would ``definitely recommend [the book], as it is a really well developed coming of age story''.

An excerpt from Benedict Randall Shaw's Piano Sonata No. 1 in C Major «L'algorithmique» concludes this issue.

The Editor would like to specially thank Benedict Randall Shaw for his assistance in the preparation of this issue.

\subsection{General notice}

\textit{The Librarian} is typeset in \LaTeX, with Scribus used to create the front and back pages, in 11pt Computer Modern.

The Editor would also like to remind readers that he accepts letters, agony aunt questions and game submissions, as well as articles.
