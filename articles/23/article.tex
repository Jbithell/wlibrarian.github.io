\textbf{Compiled by Jonny Heywood, Chairperson, and edited by Joshua Loo}

\subsection{Bookbinding comes to the Library}

This month the school bookbinders set up their annual bookbinding exhibition up Library which is still on display in the Brock Room. Featuring dozens of items, from novels and notebooks to delicately decorated symbols and letters, the display celebrates the highlights of a year’s diligent binding in Weston’s basement as well as introducing passers-by in the library to what one binder called the ‘art of the bind’. Highlights include Dominic Brind’s binding of a book on Charles V with a “golden fleece”, Captain Emeritus of Debating, Senior Binder Alfred Murray’s binding of Vita Nuova with ethereal angels on a navy leather backdrop, and the Dean's Presentation Binding, in the style of formal abbey dress, the first of Westminster's unique "cassock bindings", also Alfred Murray’s work.

On one visit, Alfred was there totting up numbers of books bound by each house (as I left it, Dryden’s led Purcell’s, then College), calculating who would win the annual ‘House Bookbinding Competition’, the winner being the house which has bound the most books since the start of the year. Alfred is the Head Bookbinder, and has an imposing collection of books on display numbering more than any other student, representing an impressive juggling act with A-Levels.

Benedict Randall Shaw, long-time contributor to the Librarian, and avid bookbinder, when asked about bookbinding, replied that “It’s a very complex process. To put it simply, one first splits the book into its composite sections, before sewing them onto cords or tape, and often rounding them for aesthetic purposes. One then sews a 'headband' on, and puts boards on the ends of the book, which one finishes with leather and/or decorated paper.” Bookbinding is a fine art indeed. “The camaraderie of the bookbinding room alone makes it a worthwhile hobby – it’s a great atmosphere”. This is a great exhibition too – well worth spending 10 minutes looking at.

\subsection{Refugee Week in the Library}

Last week the school celebrated refugee week, and the Library celebrated it in its own way. The Librarians set up a new display in the Lobby, featuring many eye-opening reads. The Good Immigrant, at the centre of the display, is a superb book satirising our impressions of what an immigrant should be, even more relevant in an age of anti-immigrant backlash. The display is an excellent collection of eye-opening books from very different perspectives, which would fit very well in any Summer reading list.

Also in the Library this week, the new book of the week on display is Reni Eddo-Lodge’s ‘Why I'm no longer talking to white people about race’, a much-anticipated book after Eddo-Lodge’s original blog post of the same name. The book deals with very profound and complex issues about intersectionality, the nature of black feminism, and an ideological struggle over the identity of feminism. Even if one weren’t to read it, the Guardian Podcast with Eddo-Lodge herself about the book and the question of identity is an excellent listen.

\subsection{Library goes silent}

For those of our avid readers who weren’t already aware, all rooms right of the lobby, including the Brock Room, Drawing Room, Christie Room and Periodicals Room, are silent areas until the end of term, to help those in exams with revision and focus.
