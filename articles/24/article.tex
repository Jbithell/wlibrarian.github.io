Welcome to our first edition of The Librarian this year, and we hope you enjoy it! The Librarian is The Library Committee’s monthly magazine, which you can find by the water dispenser in the Library, or by the entrance of the Library or up (most) houses (or wherever you got this copy from).

The Library is the one of the oldest parts of the school, and as libraries go, it’s a pretty impressive one. It is laced with history, being around in one form or another for hundreds of years. In the halls that we walk and work in, hundreds of years ago, librarians of the King kept the hugely historically important and significant artefacts and books that later formed made up British Library.

Even before the King’s Library, the very walls are steeped in history. If you walk to the back of the Library, the Greene Room and the Browne Room, or the vaulting ceilings of our 17th Century staircase tucked away in a corner showcase its restoration-era heritage. We share our library not just with each other, but with a rich architectural heritage and huge historical gravitas.

We are part of that heritage too, in a way. Nowadays, the Library is the hub of Westminster, and one with many different faces. For some, it’s a study resource, laden with information, key writings and vital texts. For others, it’s a quiet, productive space, with a studious atmosphere very conducive to hard work and vital revision. For many more, it’s a social space, where people can relax on a multitude of chairs and beanbags in what is a communal area for the entire school.

The Library is a part of every student’s life, no matter which subjects they take, which house they are in, which year they are in, and so it’s a huge honour to lead the Library Committee, a body where students can have their say in the running of the Library and where members can help with all the vital work the librarians and the Library do, from charity sales and helping students to sorting books.

It is hard to understate the role that the librarians do for us as pupils, and it weighs on me every time I enter the Library how difficult and vast a task it must be to look after the tens of thousands of books the school has, to organise and sort them into such a huge and useful tool for all students, and when one is in the Library Committee doing a fraction of their jobs, one really understands how thankful we should all be.

If you are new to the school, or if you haven’t really used the Library much before, use the opportunity, because it’s an incredible privilege to have such a historic, large, well-kept library and space, an opportunity many can only dream of having. For those who are new to the school, or reading this magazine for the first time, be sure to pick up your copy every month, and if you want to contribute, get in contact with Joshua Loo – and enjoy!

\subsection{An Invitation to the Library Committee}

The Library Committee is where pupils can get involved with the running of the Library. This means suggesting changes, talking with other pupils to find out what we could be doing better to make the Library more useful and accessible as a resource and more enjoyable a space for pupils not just of the upper years but throughout the school.

Members of the Library Committee also help as pupil librarians, setting aside some time in their busy weeks to work at the front desk, putting back books, helping pupils use the Library, and having what I find is a very enjoyable, quiet and productive space of time whilst making yourself very useful.

But the role of a pupil librarian, and the Library Committee, goes far deeper than that. With the great privilege of enjoying one of the finest and most historic libraries in Britain comes great responsibility to give back, and so at the Library Committee we help the Librarians in organising charity events and sales, the proceeds of which all go to charity.

The Library Committee also maintains \textit{The Librarian}, which is an exciting and forward thinking publication, taking contributions from both committee members and other members of the school community and weaving it into a magazine with very wide scope and high standards. If you have anything you’d like to contribute, email the editor at joshua.loo@westminster.org.uk.

If you are interested in joining the Library Committee, email me at jonny.heywood@westminster.org.uk, or just ask me in person, but remember, the role of a pupil librarian is one which requires a good deal of free time, a lot of patience, and obviously, a strong devotion to the Library. I hope to see some new faces soon!

Jonny Heywood, Chair

\subsection{Using the library}

The library may seem a daunting place at first. However, it is not actually very difficult to use. Books can be located at \url{https://oliver.westminster.org.uk} - use your Westminster login. If we do not have a book, you may suggest a book to the librarians by filling in a short form available in the library. Alternatively, speak to a librarian yourself. The librarians are always open to suggestions and very often accept them.

To find a book, first find its identifier on the Oliver search page. The first part of the identifier will be a number, eg. ``192''. In the front room of the library on the right of the doorframe to the left, there is a small sign which tells you which room the number will be in, and where the room is. The second part of the identifier is useful once you have found the room. It may simply be the first part of an author, eg. ``MIL'', or it may be another number. In any case, bookcases will have signage indicating which books are contained beneath them. They may indicate a range of numbers, or a range of letters.

The library has a number of other resources which are available on the library's Firefly page - \url{http://firefly.westminster.org.uk/library} or Firefly > Resources > Departments > Library. Online resources are contained on the eResources page. Particularly useful ones include JSTOR (good for sixth form especially), Naxos Music Library and the access to the archives and present articles of a nubmer of periodicals. The library itself receives many periodicals from across the world very frequently.

\section{Editor's note}

First, I should like to reiterate everything in the Chair's welcome. The Chair works very hard in his work with the Committee and the library as a whole.

I shall start with a few words on \textit{The Librarian}. We do not have any fixed subject. Rather, we publish anything submitted that we find intellectually stimulating. Articles must meet no requirements, save that they be intellectually stimulating and (negotiably) be less than 10,000 words. Importantly, \textit{The Librarian} \textit{does not represent the views of the Library Committee unless otherwise stated}.

\textit{The Librarian} also publishes \textit{Library News}, a one sheet publication containing library news, games, an agony aunt column and other such content.

This edition of \textit{The Librarian} has been redesigned, as our old readers may notice. This is largely due to inspiration from Felix O'Mahony, who supererogotarily submitted a design for the front page after submissions were sought for a favicon. The font size has also been increased from 10pt to 12pt, after a request by the Committee.

Adventures in Recreational Mathematics has taken an informatic turn, thanks to a stimulating submission from Benedict Randall Shaw. Isky Mathews wrote the past three adventures, and both form the team who publish it.

The cover photo is an image of Ataturk from Wikimedia Commons.

\textit{The Librarian} is now split into three primary sections:
\begin{enumerate}
	\item Dialectic,
	\item Review, and
	\item Sciences and Mathematics.
\end{enumerate}

\textit{The Librarian} is typeset in \LaTeX, with Scribus used to create the front page, in 12pt Lato and Computer Modern for mathematics.

The Editor would also like to remind readers that he accepts letters, agony aunt questions and game submissions.

To our old readers, welcome back! To our new readers, welcome!
