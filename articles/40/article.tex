The New Year has prompted a little æsthetic reflexion, culminating in the
various changes present in this supplementary. It is desired that this issue
should represent an improvement on the previous editions not only æsthetically
but also in legibility, clarity and economy---of ink, paper and so on.

It behoves me (as editor) to thank those who have supported us in our experiment
to this point: the members of the Library Committee, who have been a constant
support, the ever-present librarians---Mrs. Goetzee, Ms. Stone, Ms Stringer and
M\textsuperscript{me} Dessouroux---who have caught many typos and other stylistic problems, our
perennial scientific-cum-mathematic authors---Benedict Randall Shaw and Isky
Mathews---and other contributors---James Bithell, Benedict Mee, Thomas Adamo
and Luke Dunne to name a few, the Chair of the Library Committee---Jonny
Heywood---whose drawings on the cover have prevented several delays, and our
readers, whose existence some authors were not entirely sure of initially, but
who have surprised us by their appearance and comments in the most unlikely of
places---in the Library Committee, when walking around, and even in the lunch
queue.

It may also be useful to elaborate further on the nature of this experiment.
\textit{The Librarian} exists to publish `good' articles, for a certain value
thereof; perhaps the best description of what we seek is from Hacker News, an
online forum administered by Y Combinator, a startup incubator, who ask that
submissions should `gratif[y] one's intellectual curiosity'\footcite{hn}.
There is little further purpose. We do not desire particularly for a large
readership; we recognise that our aim is not such that all will find our
publication interesting, and are happy to leave them to read those publications
which they find more to their taste; this is, after all, an experiment.
Many have suggested that our readership is excessively small. Those who
are significantly involved in \textit{The Librarian} too often thought that
there were few readers. We have been greatly heartened by our discovery of
multiple readers; we deduce from the gradual disappearance of the copies of
\textit{The Librarian} that at least twelve or so people must read this.

Nevertheless, we have been less heartened by the response to our appeals for
contributions; we are not so much disheartened by a lack of articles or content
---though we are not exactly inundated by proposals for long form articles of
the highest quality whose publication must unfortunately be delayed---but by a
lack of other cogent perspectives; all but one of our polemics was written
by the editor, all articles relating to mathematics and the sciences were
written either by Benedict Randall Shaw or Isky Mathews, and the situation in
our review section is only a little better. There is no use in the pretence that
those who have written for \textit{The Librarian} represent a diverse spectrum
of anything, really. Certainly, we do not represent a wide spectrum of
life experiences---which, however hard we may try, will inevitably occasionally
affect one's writing, nor can it be said that we are intellectually, politically
or socially detached from each other. There may be fault on our part---it has
been suggested that \textit{The Librarian} does not appear accessible---but,
equally, there is no remedy save an increase in the
submission of articles. The present surplus of scientific and mathematic
articles, though regrettable, is simply a product of the greater advancement of
the lack of articles in other areas.

The reader need not worry, of course, that this publication will fold; there
are enough of us to steer the ship into greener pastures, though perhaps not
enough to avoid Jim Hacker-like mixing of metaphors and their analogues elsewhere.
Those more prone to worries of civilisational collapse or general qualitative
collapse in the constitutions of the present generation would not, however, be
swayed from such views were they to read some of the older editions of
\textit{The Elizabethan}, not so much because they would find its content
better---for the more serious content has not disappeared but moved to other
publications, viz. \textit{Camden} and \textit{Hooke}---but that there seem to
have existed a vast array of pupil-run independent publications. That it
happened that the population of a house---in a time where the conveniences
of modern technology were scarcely known to the typewriter-employing creatures
of the era---was able to sustain an entire termly publication is astonishing.
That there were several mentioned, viz. the \textit{Grantite Review},
\textit{Ashtree}, \textit{Rigaudite Review} and \textit{College Street
	Clarion}, and that there was, further, a scientific publication, viz.
\textit{Nucleus}, a cultural publication, viz.
\textit{Polygon}\footcite{magazines} and yet \textit{The Elizabethan} did not
appear to suffer from the want of articles later complained of by the
1970s\footcite{elizabethan76} could perhaps most charitably be described as an
indictment of the relative state of the present pupil body.

One should not be surprised to find that our proposed remedy is writing for
\textit{The Librarian} (and, indeed, all other school publications); the present
crop of publications is published very irregularly---\textit{The Librarian}
seems to be the most frequent, but is rather short, and the others are far too
infrequent. The problem is not so much that there is a lack of publications, so
much as that they are too infrequently published to make a major impression upon
school life, and will only be rectified if we should contrive to increase the frequency of their publication.

Some readers may ask why one should wish for pupil publications\footnote{It
would,however, be rather surprising to discover that one such
as this had decided to read \textit{The Librarian}.} at all. To them, we say that a
publication ensures that those who seek to write are able to embarrass themselves
before they reach the outside world; it ameliorates the fluency and clarity with
which those who contribute and those who read think; it provides an historical
record for those who will in the future wonder how we write, think and spend
our time at present; it is, in short, a great contribution to the denizens of
the community whose time it disturbs or graces. We should find no
