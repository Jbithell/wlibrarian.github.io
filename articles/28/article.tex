At 6am on the 30th December 2006 Saddam Hussein was hanged in a joint Iraqi-American military base known as “Camp Justice”, in a suburb just north of Bagdad [1]. Saddam had been the leader of Iraq for 23 years, but after a new government was installed by an international coalition led by President George Bush in 2003, he was executed in a manner the Bush said “looked like it was kind of a revenge killing.” [2]. Does Camp Justice live up to its name? Was it right for the new government of Iraq to decide to pursue the execution of their predecessor?

Politicians are by definition the most accountable people in our society. Judged by all the people that make up a nation, not just a small cross section through a jury, national politicians set our laws, dictate how people should be tried and how they should be punished. A head of state is, according to the common legal maxim \textit{rex non potest peccare} (the king can do no wrong) [3] [4], above the law and cannot be tried in most courts. Thanks to the ideas of state sovereignty enshrined in international law, a head of state can not be tried in a foreign court either, and they are therefore whilst in office essentially “above the law”. This is significant for monarchies where a blood line dictates the line of succession and the office is only given up on death, such as is the case in Britain. In this situation there is no way of bringing legal action against the head of state as they spend a lifetime in office, whilst for many presidents and other types of heads of state, prosecution is often brought against them after their departure from office, albeit often unsuccessfully, as sometimes the “office” takes a decision, not the individual that holds the position of office. Whilst this is problematic - Romania’s parliament spent a great deal of 2015 refusing to lift the parliamentary immunity of an MP who was accused of taking bribes [5], we shall only discuss the prosecution of former politicians here, and ignore the complications of prosecuting someone in office (and mechanisms for removal from office – such as impeachment). We shall also mostly exclude offences where individuals are always almost prosecuted and imprisoned – this essay does not seek to discuss whether politicians should be excused from prosecution for rape or murder.

Simply put, individuals are prosecuted to protect the public from them should they be considered a risk (and to prevent further offences through various measures such as imprisonment), to provide closure and vindication to the victim or their families, and to serve as a deterrent to others [6]. In turn, this piece consider the application of each to the prosecution of former politicians.

It’s illegal for a car manufacturer to advertise functionality their cars can not actually fulfil [7]– but not illegal for a politician to make a promise they cannot fulfil. Why is it illegal for the car manufacturer to lie? Simply put, “to protect the public”. So should a politician who lies therefore be prosecuted for a new crime such as “lying to gain office”, or is this just a waste of the law’s time? Politifact, a US based fact checking website, runs a “truth-o-meter” where they evaluate statements made by politicians for truthfulness. During the 2017 general election campaign they rated 307 statements made by Republican party candidate Donald Trump either “Mostly False”, “False” or “Pants on Fire”, but only 77 for Democratic party candidate Hilary Clinton [8]– clearly these falsehoods being drawn attention to does not work. Must we therefore prosecute? This is something argued by Marcus, a 27-year-old from Norwich [9], who has lead a crowdfunded campaign to bring a private prosecution to court for politicians who lied during the Brexit campaign, and to “establish a legal precedent in UK common law that prevents political leaders from lying to the public in future” [10] – much like the “lying to gain office” offence. He has raised over £145,000 [11] so far for research and case building, and is now trying to raise another £2 million to fund the lengthy court battle. But perhaps it is not that simple – many argue that part of the central underpinning of a democracy is that the public should be able to choose who they wish to represent them in making decisions.  John Stuart Mill, in his 1859 \textit{Essay on Liberty} and 1861 \textit{Considerations on Representative Government}, argued that every adult should have the vote but only after compulsory secondary education [12]. Today with universal secondary education in the UK Bernard Crick, writer of \textit{Democracy, a Very Sort Introduction}, argues that “[p]olitical leaders can cry ‘education education education’, but with their manipulation of the media, sound-bites, and emotive slogans rather than reasoned public debate, Mill might have had difficulty recognising them as products of an educated democracy” [12]. So perhaps the public should be legally protected from their own impulse? It is fair to say that any legislation that could be perceived to “dumb down” politics would be hopelessly unpopular, but if we had to protect the public from themselves, how would we do that with the problem of how easy it would be for an incoming government to prosecute the previous government under the guise of protecting the public from those in the previous administration, in doing so removing threats to themselves. Perhaps this is partly what happens during inquests organised by certain UK governments. In June 2009, the then prime minister, Gordon Brown, announced a public inquiry into the actions of the previous government leading up to, and during, the war in Iraq in 2003 [13]. The inquiry had no authority to prosecute anyone, and thus it is easy to think that it is not relevant to this question, but this is where the politics comes in. For senior figures at the time the result of this inquiry could be described as a fate worse than prosecution. A prosecution and conviction had not been ruled out, but the political fallout has been just as bad – if not worse. In the case of Tony Blair, Brown’s predecessor, the Guardian’s Richard Norton-Taylor summed it up: “Chilcot's indictment of Tony Blair could hardly be more damning” [14]. So perhaps we do not need to prosecute former politicians – maybe just public justice through an inquiry is good enough? Or is this just a way of the next administration trying to crush the previous one? And what about incidents that do not warrant a costly public inquiry such as lying during an interview? We shall come back to these questions later on.

Prosecutions are also led against individuals to provide closure and vindication to the victims of crimes, so how do we apply that to the prosecution of a politician? It is easy to say that the crimes we are discussing are victimless – who is a real victim from a politician lying as in the example above? Surely it would be a waste of time to prosecute politicians? What if a politician took a poor decision or lied to the public before going to war? Causing death by careless driving is not a victimless crime - it has real victims and their families, and results from poor decisions taken by an individual. So would not it be a good idea to prosecute a politician who has made a poor decision? There are two issues here – one is the faith that that decision was taken in, and the other is whether that would make all future leaders risk averse. The high court has blocked attempts to bring a private war crimes prosecution against Tony Blair over his role in the Iraq invasion of 2003. Though charged with the crime of aggression under international law by the private prosecution, this does not apply retroactively in British law and therefore is not applicable to the Iraq Invasion [15] [16]. In British terms, there were 182 casualties resulting from the Iraq invasion [17] – they and their families are all real victims. But one could not prosecute politicians for an entire war, surely? Would you prosecute Winston Churchill for the 449,700 British killed in the Second World War [18]? After all, the servicemen killed during that war were mostly conscripted, not the voluntary army that fought in Iraq. Though it may be appropriate to prosecute individuals over specific failings, such as the crash of a military aircraft due to poor maintenance, as the owner of an airline would be prosecuted for a similar incident to one of their planes, we should stop short of prosecution for crimes of “aggression”, such as an invasion Tony Blair said was taken in “good faith”, but that turned out to be more complex due to the unforeseen threat of insurgency [19]. Sir John Chilcot, who chaired the inquiry into the Iraq war, disputes this, and said that he does not believe Blair was “straight with the nation” [20]. That is enough -  there was a seven year investigation [21], which was well reported and published with the lessons learned. For the families of those killed, the terrible tragedy will stay with them forever, whether those in charge are prosecuted or not.

Christopher Paul-Huhne, as he was known whilst a pupil at Westminster School, was a senior liberal democrat politician, serving as Secretary of State for Energy and Climate Change during the coalition government from 2010-2015 [22]. Caught speeding in his BMW on the 12th March 2003 [23], he asked his wife Vicky to take his penalty points to prevent him being banned from driving. When this came to light in 2013 and he was charged for perverting the course of justice, becoming the first British cabinet minister in history to resign because of criminal charges brought against him [24]. Following a guilty plea he was convicted and jailed. In October 2011 the AA ran a survey, the results of which revealed that up to 300,000 drivers may have persuaded others to take their penalty points for speeding [25], though only 12\% of AA members said they would refuse and make a report to the police if asked to participate in a point swap. Only around 12,000 people a year are convicted for what the UK’s Crown Prosecution Service defines as “other offences”, of which perverting the course of justice is one [26] – so even if every one of those convictions was for perverting the course of justice (which they are not) only 4\% of those estimated to be “point swapping” would be being charged. We know the actual conviction figure is far far lower than this because that category statistic includes a swathe of other crimes. But it is very difficult for the police to detect it happening, and so it is fair to say that the fact few are convicted is not relevant to a discussion of Huhne – and none of this suggests Huhne was singled out. The Daily Mail’s Sarah Williams, writing in May 2011, discussed her conviction in court for perverting the course of justice [27] after her mother agreed to pretend she was driving the car in which she was caught speeding, but the police were tipped off. She was fined £1,000, handed a six month driving ban and 125 hour community service order. Huhne was imprisoned for eight months. So why the difference? Why was discretion used for Williams but not Huhne? This comes down to our final reason to prosecute, to serve as a deterrent. Huhne’s prosecution attracted huge media attention, brought stories like Sarah’s to light, and led to surveys such as the one conducted by the AA previously discussed. So prosecuting politicians more than the general population to set an example is right – but is it fair? Well perhaps– politicians agree to uphold the law in taking public office, and in doing so accept that they will undergo more scrutiny as a result of more media attention. There is not a great deal the law could or should do about that - “public vetting” is a big part of choosing who to vote for in a democracy.

To conclude, though this essay has raised just as many questions as it has attempted to answer, hopefully this piece has provided a relatively balanced argument for why it may or may not be right to prosecute former politicians, leaning towards a conclusion that it is not right to prosecute former politicians for perceived “lying” to the public or for “aggression” which was part of an attempt to protect the people of a foreign nation in an intervention that was misjudged. The current mechanism of public inquiries provides sufficient framework for determining the facts from those sorts of incidents and learning from them – which is the most important thing. For politicians that lie, perhaps in a TV interview, it is clear that the public of the “information age” [28] are able to discover what’s true and what is not, but either choose to ignore these “lies” or do not believe they are significant enough to influence their decision making. Legislating to “protect the public from themselves” is a dangerous game and it is easy to see how a newly elected government could attempt to annihilate their opponents by prosecuting them for perceived crimes – something we have not seen on a large scale since Stalin’s USSR. Politicians should be held to account by our vote – it is our role as the people of a democracy to use the tools at our disposal to make the right decisions. We shall make mistakes, and so will our leaders, but we do not need to prosecute them for it

\subsection{Bibliography}
\begin{enumerate}
	\item "BBC NEWS | Middle East | Saddam Hussein executed in Iraq," , . [Online]. Available: http://news.bbc.co.uk/2/hi/middle{\textunderscore}east/6218485.stm. [Accessed 6 8 2017].
	\item PBS, "PBS News," 16 January 2007. [Online]. Available: http://www.pbs.org/newshour/bb/white{\textunderscore}house/jan-june07/bush{\textunderscore}01-16.html. [Accessed 17 January 2007].
	\item "Sovereign immunity," , . [Online]. Available: http://en.wikipedia.org/wiki/Sovereign{\textunderscore}immunity. [Accessed 20 8 2017].
	\item Bithell Studios, "Sapiens Optio - Latin Translator," [Online]. Available: https://sapiensoptio.com/.
	\item The Economist, "Why politicians are granted immunity from prosecution," 27 5 2016. [Online]. Available: https://www.economist.com/blogs/economist-explains/2016/05/economist-explains-21. [Accessed 30 8 2017].
	\item British Broadcasting Corporation Archive, "Ethics," [Online]. Available: http://www.bbc.co.uk/ethics/capitalpunishment/for{\textunderscore}1.shtml. [Accessed 25 8 2017].
	\item Her Majesty's Government , "Marketing and advertising: the law," [Online]. Available: https://www.gov.uk/marketing-advertising-law/regulations-that-affect-advertising. [Accessed 30 8 2017].
	\item PolitiFact, "Comparing Hillary Clinton, Donald Trump on the Truth-O-Meter," [Online]. Available: http://www.politifact.com/truth-o-meter/lists/people/comparing-hillary-clinton-donald-trump-truth-o-met/. [Accessed 30 8 2017].
	\item The Independent , "Brexit: Crowd-funded legal case wants to see politicians jailed for 'lying' during referendum campaign," 14 10 2016. [Online]. Available: http://www.independent.co.uk/news/uk/politics/brexit-latest-eu-referendum-brexit-justice-legal-case-dishonest-lying-politicians-a7362096.html. [Accessed 30 08 2017].
	\item M. J. Ball, "{\#}BrexitJusitce," [Online]. Available: http://www.brexitjustice.com/. [Accessed 30 8 2017].
	\item Crowd Funder, "{\#}BrexitJustice," 29 7 2016. [Online]. Available: http://www.crowdfunder.co.uk/brexitjustice. [Accessed 30 8 2017].
	\item B. Crick, Democracy : a very short introduction., Oxford New York: Oxford University Press., 2002.
	\item Wikimedia Foundation, "Wikipedia," [Online]. Available: https://en.wikipedia.org/wiki/Iraq{\textunderscore}Inquiry{\#}History. [Accessed 28 8 2017].
	\item The Guardian, "Chilcot's indictment of Tony Blair could hardly be more damning," 6 7 2016. [Online]. Available: https://www.theguardian.com/uk-news/2016/jul/06/chilcot-indictment-of-tony-blair-could-hardly-have-been-more-serious. [Accessed 28 08 2017].
	\item British Broadcasting Corporation, "Iraq War: Bid to prosecute Tony Blair rejected by High Court," 31 7 2017. [Online]. Available: http://www.bbc.co.uk/news/uk-40775725. [Accessed 30 8 2017].
	\item The Guardian, "Tony Blair prosecution over Iraq war blocked by judges," 31 7 2017. [Online]. Available: https://www.theguardian.com/politics/2017/jul/31/tony-blair-prosecution-over-iraq-war-blocked-by-judges. [Accessed 30 8 2017].
	\item Wikimedia Foundation, "British fatalities during Operation Telic," [Online]. Available: https://en.wikipedia.org/wiki/British{\textunderscore}fatalities{\textunderscore}during{\textunderscore}Operation{\textunderscore}Telic. [Accessed 30 8 2017].
	\item Wikimedia Foundation, "United Kingdom casualties of war," [Online]. Available: https://en.wikipedia.org/wiki/United{\textunderscore}Kingdom{\textunderscore}casualties{\textunderscore}of{\textunderscore}war. [Accessed 30 8 2017].
	\item T. Blair, A Journey, London: Penguin Books, 2011.
	\item British Broadcasting Corporation, "Iraq Inquiry: Full transcript of Sir John Chilcot's BBC interview," 6 7 2017. [Online]. Available: http://www.bbc.co.uk/news/uk-politics-40510539. [Accessed 30 8 2017].
	\item Wikimedia Foundation, "Iraq Inquiry," [Online]. Available: https://en.wikipedia.org/wiki/Iraq{\textunderscore}Inquiry. [Accessed 30 8 2017].
	\item Wikimedia Foundation, "Wikipedia - Regina v Christopher Huhne and Vasiliki Pryce," [Online]. Available: https://en.wikipedia.org/wiki/R{\textunderscore}v{\textunderscore}Huhne. [Accessed 29 8 2017].
	\item British Broadcasting Corporation, "BBC News - Chris Huhne and Vicky Pryce jailed for eight months," [Online]. Available: http://www.bbc.co.uk/news/uk-21737627.
	\item The Daily Mail, "Millionaire minister Chris Huhne, forced from office by criminal prosecution, wants £17k pay-off from the taxpayer," [Online]. Available: http://www.dailymail.co.uk/news/article-2107901/Chris-Huhne-Minister-facing-trial-swapping-speeding-points-demands-17k-severance-pay.html. [Accessed 25 8 2017].
	\item The AA, "News - Swapping penalty points," 13 10 2011. [Online]. Available: http://www.theaa.com/motoring{\textunderscore}advice/news/points-swapping.html. [Accessed 30 08 2017].
	\item Crown Prosecution Service , "Case Outcomes," 07 08 2017. [Online]. Available: http://cps.gov.uk/data/case{\textunderscore}outcomes/index.html. [Accessed 30 08 2017].
	\item The Daily Mail, "Would you ask a loved one to take your speeding points? I did and I still live with the consequences," 25 5 2011. [Online]. Available: http://www.dailymail.co.uk/femail/article-1390586/Would-ask-loved-speeding-points-I-did-I-live-consequences.html. [Accessed 30 8 2017].
	\item Wikimedia Foundation, "Information Age," [Online]. Available: https://en.wikipedia.org/wiki/Information{\textunderscore}Age. [Accessed 30 8 2017].
\end{enumerate}
