Hume once wrote:

\begin{displayquote}

	In every system of morality, which I have hitherto met with, I have
	always remarked, that the author proceeds for some time in the ordinary
	way of reasoning, and establishes the being of a God, or makes
	observations concerning human affairs; when of a sudden I am surprised
	to find, that instead of the usual copulations of propositions, is, and
	is not, I meet with no proposition that is not connected with an ought,
	or an ought not. This change is imperceptible; but is, however, of the
	last consequence. For as this ought, or ought not, expresses some new
	relation or affirmation, 'tis necessary that it should be observed and
	explained; and at the same time that a reason should be given, for what
	seems altogether inconceivable, how this new relation can be a deduction
	from others, which are entirely different from it. But as authors do not
	commonly use this precaution, I shall presume to recommend it to the
	readers; and am persuaded, that this small attention would subvert all
	the vulgar systems of morality, and let us see, that the distinction of
	vice and virtue is not founded merely on the relations of objects, nor
	is perceived by reason.

\end{displayquote}

What Hume meant was that an argument must have at least one moral
premise before it can logically make a claim that an action is wrong
based on factual analysis. An example may clarify this. That an action
would cause deaths does not necessarily mean that it is bad - one must
also claim that death is bad.

\textit{The Librarian} has hitherto been split into a dialectic and
scientific-cum-mathematic section. The latter's name is potentially
inadequate; the humanities make no claim to be sciences, but
nevertheless do not always prescribe - often they merely describe. Hence
the scientific-cum-mathematic section will be renamed ``Is'', and
dialectic ``Ought''.

In the coming weeks, we shall, hopefully, smooth our workflow, so as
to enable issues to be generated faster and on time. This will include
our weekly \textit{Library News} publication, which has not quite met its
publication schedule to a combination of factors.

Today's issue commences with the Editor's (now finished) article on the death of Chinese pluralism. It continues with Benedict Randall Shaw on graph theory, before reviewing lectures and \textit{American Classics}.

On the cover is Jonny Heywood's sketch of the chair which Liu Xiaobo was to sit in to receive his Nobel Prize before the Chinese government prohibited his leaving the country. Liu died on 13 July 2017 after a battle with cancer; his condition was exacerbated by his incarceration and inhumane treatment.



\subsection{General notice}

\textit{The Librarian} is typeset in \LaTeX, with Scribus used to create the front and back pages, in 11pt.

The Editor would also like to remind readers that he accepts letters, agony aunt questions and game submissions, as well as articles.
