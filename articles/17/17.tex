\documentclass[10pt,a4paper]{article}
\usepackage[utf8]{inputenc}
\usepackage[margin=1.5cm,includehead,includefoot]{geometry}
\usepackage{enumitem}
\usepackage{fancyhdr}
\usepackage{multicol}

\pagestyle{fancy}
\fancyhf{}
\fancyhead[R]{SUBJECT}
\fancyhead[L]{Joshua Loo}
% \fancyfoot[C]{\thepage}

\setlength{\parindent}{0em}
\setlength{\parskip}{1.2em}

\renewcommand{\footrulewidth}{0pt}

\newcounter{count}

\begin{document}
\setcounter{section}{-1}
\title{Crooke's \textit{Resistance}}
\date{4 September 2017}
\author{Joshua Loo}

\maketitle

\begin{multicols}{2}
	
	Discourse about Islam often blithely assumes Islamic inferiority,
	whether in overtly advocating its suppression or appeasement.
	\textit{Resistance}'s dual account of Islamic intellectual history from
	the breaking up of the old Ottoman order to the modern structure of
	Hamas and Hezbollah and the religious roots of the instrumental
	rationality of Western orthodoxy is no less relevant today than they
	were at their publication in 2009.
	
	Crooke's account of the imposition of westernisation is not particularly
	novel, but it is written well enough, and is quite passable. His
	critique of Western instrumental rationality, though somewhat incomplete
	and lacking in philosophical depth, is far from inaccessible, and
	contains a number of useful insights. Again, this account is not
	completely novel, but it is still adequate.
	
	The granting of equal status to Islamist ideology enables an interesting
	and perhaps novel comparison with the Western intellectual canon. The
	book shines when it questions the distinction often drawn between
	Western intervention and Islamist jihad. Its praise of Islamism as an
	open ideology, and the distinction drawn in action and philosophy
	between the eschatology of Al Qaida and the ``emancipatory resistance''
	of Hamas and Hezbollah is another highlight. \textit{Resistance} contrasts
	this with the religious roots of Western just war theory.
	
	The book is moderately well written. It has a relatively wide frame of
	reference, as would be expected of a book with its thesis. Occasionally
	it is a little difficult to follow due to unintuitive sentence
	structure, but that itself is no major crime. Although accessibly
	written, it has not obviously reduced the complexity of the ideas that
	it presents.
	
	However, Crooke ignores a number of issues which would complete and
	round his account. Though Western ideology may have its flaws as he
	posits, he ignores a number of problematic views and actions in these
	groups. Consider that Hamas's 1988 charter, which is still in force,
	mentions a ``struggle against the Jews'' and approvingly quotes:
	
	\begin{quote}
		"The Day of Judgement will not come about until Moslems fight the Jews
		(killing the Jews), when the Jew will hide behind stones and trees. The
		stones and trees will say O Moslems, O Abdulla, there is a Jew behind
		me, come and kill him. Only the Gharkad tree, (evidently a certain kind
		of tree) would not do that because it is one of the trees of the
		Jews."\footnote{http://avalon.law.yale.edu/20th\_century/hamas.asp}
	\end{quote}
	
	Any attempt to legitimise such movements must also account for this.
	
	The Middle East Media Research Institute, an organisation which has been
	accused of deliberate mistranslation or misleading translation in the
	past, but whose translations are rarely completely incorrect, reported
	that the Secretary-General of Hezbollah declared that homosexual
	relationships were ``contrary to logic, human nature and the human
	mind''\footnote{https://www.memri.org/tv/hizbullah-sec-gen-nasrallah-warns-against-legalization-gay-marriage-lebanon-defends-early/transcript}.
	
	These issues may not be as large has is otherwise suggested, but they
	still ought to have been addressed.
	
	Nevertheless, it is still well worth reading.
	
	\textit{Resistance: the essence of the Islamist revolution} is available
	in the library, or will be when the Editor has returned it.
	
\end{multicols}

\end{document}
