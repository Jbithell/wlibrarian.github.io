\documentclass[10pt,a4paper]{article}
\usepackage[utf8]{inputenc}
\usepackage[margin=1.5cm,includehead,includefoot]{geometry}
\usepackage{enumitem}
\usepackage{fancyhdr}
\usepackage{multicol}

\pagestyle{fancy}
\fancyhf{}
\fancyhead[R]{On common institutions and dialectic speech}
\fancyhead[L]{Joshua Loo}
\fancyfoot[C]{\thepage}

\setlength{\parindent}{0em}
\setlength{\parskip}{1.2em}

\renewcommand{\footrulewidth}{0.5pt}

\newcounter{count}

\begin{document}

\title{On common institutions and dialectic speech}
\date{4 September 2017}
\author{Joshua Loo}
\maketitle

\begin{multicols}{2}
	
	The following note has been added for clarity.
	
	\textbf{This article is not published in my capacity as the Editor of \textit{The Librarian}, nor does it represent the views of the library or the Library Committee. It is published in my personal capacity. The headmaster has discussed this article with me and raised no objection to its publication or specific objection to its content. The title of this article has been changed so as to shift emphasis onto a consideration of the general, not special, case. It is published here for historical and archival purposes.
	}
	
	When the headmaster fatuously attacked protectionism as a false God, he
	appeared to commit three sins. First, he exploited what ought to be a
	neutral platform, ie. that of the headmaster in a school assembly, for
	political ends. Second, there was no opportunity for reply, either in
	the form of similarly vacuous hand-wringing or actual consideration.
	Third, he failed to provide any analysis as to why his claims are true -
	a rather dubious method by which one might follow a ``Gladstonian''
	liberal tradition which presumably ostensibly embraces substantive
	dialogue rather than empty rhetoric.
	
	Yet as we explore how platforms in common institutions should be used,
	we come up against a number of barriers in coherently defining a way to
	exclude some but include others with a sufficiently nuanced and accurate
	brush; not only this, it is difficult to generalise such rules so that
	they apply from different ideological viewpoints.
	
	Before that, we should note two recent pronouncements from the headmaster:
	\begin{enumerate}
		\item The headmaster attacked Trump and protectionism in Latin prayers last term, and has repeatedly promoted his idea of Gladstonian liberalism.
		\item The headmaster claimed that the Armenian genocide was the first in the twentieth century.
	\end{enumerate}
	
	\section{Substance}\label{substance}
	
	The point of speech which departs from that which has been
	established\footnote{This is a somewhat confusing idea, which is
		difficult to define. Broadly speaking it includes what is either
		agreed with by everyone who is listening (eg. death is bad), is
		implicitly assented to even without actual agreement (eg. abbey
		service is compulsory) by some other action (eg. attending
		Westminster), is implied by anything which is already agreed upon (eg.
		that death is bad implies that Grenfell was a tragedy), or is the
		object of a broad scholarly consensus (eg. climate change is
		anthropogenic, Mathematics is true and Paris exists). An objection to
		the last statement was received in the writing process, so a
		modification to this is that this consensus can either be defined by
		the proportion of applicable persons who subscribe to a particular
		view, or in terms of the totality of the scholarly leanings of the
		community. The latter is to say that even if there is epistemic
		\textit{doubt} as to whether Paris exists, few scholars (unless
		attempting something weird semantically) will deny that Paris exists,
		and will merely attempt to induce doubt. If we total the overall
		leanings of a community where some are uncertain and others are more
		certain, all in one direction, we can say that there is some sort of
		consensus there too. This term will continue to be used throughout
		this essay. An addendum is that some of what is established is
		accepted for differing reasons; deontologists and utilitarians both
		agree that for all actions $x$, where $x$ only causes a
		death, $x$ is bad.
		
		Although these may seem intuitively true, they will each in turn be
		justified in the second section. Some of the latter subsets of
		``established'' truth are merely permissible because they form
		important subsets in turn of the first two.} is presumably to convince
	us that there are more things which should be established; in turn, this
	relies on the assumption that we should like to find truth. Given this,
	reply is important. An initial argument is not necessarily correct;
	corrections are important. Even if the original thesis of an argument
	turns out to be correct, the process of rebuttal strengthens arguments -
	knowing the same truth for better reasons is itself desirable. Finally,
	reply, and further replies in turn, increase the perception that truth
	has been found. When a good process for finding the truth has been
	found, ie. argument, it is important that other people recognise its
	validity, the more to transmit truth.
	
	Protectionism's lack of efficacy has not been established in three ways.
	First, it is unclear which moral system is best used to determine what
	is good or bad. A ``utilitarianism of money'', which blindly seeks the
	greatest gross global product, would differ in its recommendations to
	preference utilitarianism, Rawlsian justice as fairness in its original
	moral formulation, or rights theory. Second, even practising a
	Li-style\footnote{Li, S. (2017) ``Ethics and Market Design'', accessed
		via shengwu.li} ``informed neutrality between reasonable ethical
	positions'', it is not clear whether protectionism ``works'' - Ha-Joon
	Chang writes that protectionism in some forms was crucial to
	industrialisation in a number of countries\footnote{Chang, Ha-Joon
		(2002) \textit{Kicking Away the Ladder: Development Strategy in
			Historical Perspective.}}, in a tradition of correlated protectionism
	and development stretching back at least to high American tariffs in the
	19\textsuperscript{th} and early 20\textsuperscript{th} century, South
	Korea's rapid industrialisation and a number of other examples; this
	industrialisation and the concomitant reduction of human misery are
	presumably good things in all reasonable ethical positions. Third, at no
	point was it established as a founding truth for the school, nor was it
	acknowledged or assented to. We may, for example, implicitly agree not
	to be violent to each other, with all the constraints on the full
	implementation of certain ideologies this entails, in going to school;
	we do not agree to the headmaster's Gladstonian liberalism in its full
	muscular ``glory''. Hence in this sense, what the headmaster does not
	fall into the set of established ideas which are broadly acceptable.
	
	What sort of principles should be applied to those who wish to depart
	from what has been established, in terms of their presentation?
	
	First, and perhaps most importantly, they must act so as to maximise
	truth. There was little chance of anyone shouting in reply ``no it
	isn't'' to the headmaster without sanction; this is obviously a
	relatively trivial barrier to remove. Replies must be available, and
	listened to by everyone who listens to the original; if attendance is
	compulsory, attendance to the reply must also be compulsory, so that
	reply actually works. They must also be given sufficient time, and due
	notice. Additionally, any form of sanction oughtn't to exist. They may
	inevitably exist unless the headmaster is, for example, stripped of all
	power within the school, but need not unnecessarily exist. It mustn't
	necessarily be the case that everyone is afforded the opportunity to
	speak. It is however, desirable that at the very least an opposing
	viewpoint is heard.
	
	Second, and relatedly, these views mustn't be presented as established.
	A headmaster appears to hold the full force of the coercive and
	administrative apparatus of the school. It must be expressly stated that
	whatever he says is not the view of the school if it is not established
	as such. This separation must be made clear so as to enable the full
	implementation of the first principle\footnote{The theme of the effect
		of authority is continued later, in ``Neutrality: a desideratum?''.}.
	This was not done.
	
	Third, these views must be important. There is limited time and limited
	space for discussion. Importance can be defined in a moral sense - the
	imparting of important moral truths, ie. what \textit{ought}, can be said
	to be important, and we can measure importance in this way. Truths as to
	what \textit{is} are also important\footnote{This is in reference to the
		is/ought distinction; see Hume's \textit{A Treatise of Human Nature},
		Book III, Part I, Section I, p. 469. Hume notes that we cannot
		determine what ``ought'' to be the case, ie. moral truth, without at
		least one other such truth; without any, we cannot say that anything
		is bad. An example may clarify: ``$x$'s sole consequence is a
		death'' does not imply that ``$x$ is bad'' unless we also accept
		the moral premise that ``we oughtn't to murder'', or ``death is bad''.
		``Is'' statements may imply ``ought'' statements in conjunction with
		other ``ought'' statements. Though this is the case, following Li's
		model of ``informed neutrality between reasonable ethical positions'',
		we can take some ought-conditions to be given (eg. death is bad), and
		so consider things which affect lots of people, and therefore may
		interact with ought-conditions, to be important as well, though they
		only describe what is.}, if they are important based on two kinds of
	ought. The first is when an ought is already established. The second is
	when the case that something ought to be the case is made in the same
	way, as described before.
	
	The lack of desirability of the third sin is uncontroversial. Those who
	decry protectionism likely seek a better justification and presentation
	of this view than what occurred in Latin prayers. Those who support it
	would simply note the falsity of his claim. It is not particularly
	controversial to seek an amelioration in standards of discourse, though
	the headmaster is not alone in dialectic moronisation and is perhaps
	better than certain others in his ability to orally extemporise a series
	of grammatical English sentences, each with a main verb, unlike some of
	the present ``friends'' of one (now part-time) occupant of Chequers.
	
	Hence we add another two principles. Fourth and relatedly, views must be
	rigourous and have a moderate chance of success, even if they do not
	fully convince. This is partly related to the third principle, ie. that
	of prioritisation, but it is also because poorly justified views are
	more likely to be incorrect than rigourously justified views, insofar as
	ordinarily that true views are more easily justified.
	
	Fifth, nothing in these principles fails to apply to the assumptions,
	anecdotes, quotes and any other topics during all compulsorily attended
	discourse. An inadvertent denial of the Herero and Nama genocide in
	German-ruled Namibia by terming the Armenian genocide ``first'' in the
	twentieth century would be unacceptable even were it not to be a main
	message.
	
	\section{Neutrality: a
		desideratum?}\label{neutrality-a-desideratum}
	
	The solution to the first sin, whether in Liberty University's
	compulsion of students to listen to Ted Cruz\footnote{It appears that
		this also occurs with other speakers, eg. Bernie Sanders, as part of a
		weekly assembly. Liberty University is questionable in a number of
		ways but the imbalance in speakers appears to be caused by progressive
		speakers' refusal to accept invitations to speak.} or Westminster's to
	the Headmaster appears to be neutrality. Compulsory school meetings
	ought to simply be about school - fire drills, timetabling, achievements
	and other educational minutiae. This has a certain appeal - it seems to
	be functional (who gains from vacuous hand-wringing?), fair, and clear.
	
	\subsection{Is compulsorily attended dialectic discourse
		helpful?}\label{is-compulsorily-attended-dialectic-discourse-helpful}
	
	Compulsory discourse which is overtly dialectic causes three harms.
	
	First, it alienates those who disagree with the message it spreads.
	Although this may not seem like a harm in all circumstances, eg. if it
	alienates racists, the neutrality of institutions is itself a good in
	that it enables different otherwise mutually incompatible sections of
	society to cooperate. Consider, for example, rubbish collection: if
	rubbish collection is politicised (eg. if all rubbish bins were to be
	painted with anti-racist slogans), it is possible that those who
	disagree (even if they are wrong\footnote{The effect described here is
		plausibly more likely to occur when the group who disagree with a
		politicisation attempt have poorly justified views; poorly justified
		views are more likely to be grounded in some sort of blind faith,
		whether religiously or socially motivated, and so they are plausibly
		held with much more faith and strength than views which are based on
		other views and are somewhat instrumental. Contrast two opponents of
		euthanasia, the first of whom believes that it is biblically
		forbidden, the second likely to cause people to be forced into
		coercion. The former will be very hard to convince - biblical
		teachings are hard to rebut - compared to the latter, who may be
		convinced that this coercion is less important than the coercion
		experienced by those forced to live, or that it is very unlikely.})
	will stop cooperating with rubbish collection. This is bad because
	rubbish collection is environmentally important.
	
	Second, it unfairly advantages certain points of view. Arguments cannot
	be agreed with if they aren't heard. Where there is compulsory
	discourse, it is often delivered by an authority figure - a headmaster
	or a graduation speaker, to take a few examples. Listeners may view such
	discourse with undeserved respect. This is not to say that these figures
	cannot be correct, but rather that they are less likely to be correct
	than the people whom would be trusted by the force of their argument,
	unbuttressed by the perceived stature which many of these authority
	figures possess.
	
	Third, it alienates those who do not wish to listen to dialectic
	reasoning. Compulsory attendance already is sufficiently coercive to
	cause resentment. This resentment is compounded by the addition of
	seemingly unneeded dialectic speech. Why must more of their time be
	wasted? They are not listening - they are worse than idling, for they
	who must listen unwillingly to headmasterly pronouncements on
	Gladstonian liberalism are not enjoying themselves. Even if a lack of
	interest is to be decried, one's aim oughtn't to be to maximise the
	misery of the uninterested but to increase their interest. This feeling
	of resentment is entirely unnecessary; it reduces the likelihood that
	individuals will naturally gravitate towards interest, which will
	produce a superior organic and sustainable interest.
	
	This harm must be weighed against the benefit to, to continue this
	example, the people who feel supported by anti-racist slogans. The
	benefit is not only not particularly large, but also is not uniquely
	achieved by such an action. There are many other ways to promote
	anti-racism and make oppressed racial minorities feel happy. Where there
	are other avenues open which directly concern a matter\footnote{Consider
		harsher penalties for racial abuse, a privately funded publicity
		campaign or an outreach programme.}, one oughtn't to use institutions
	whose specific purpose would be negatively impacted.
	
	If there are no such avenues, such action ought to be permissible only
	if it meets two conditions.
	
	First, the harm to, for example, rubbish collection, must be minimised,
	in attempting to use the least destructive means possible, using the
	previous set of avenues to the maximum extent possible and so on. This
	ought to be fairly obvious.
	
	Second, benefit must outweigh systemic and immediate harms to, to
	continue the example, rubbish collection. We must consider all the extra
	rubbish which will be incorrectly sorted, littered or thrown into rivers
	etc.; we must consider the harm to cooperation with other council
	services as well. Insofar as this is very difficult to quantify and
	predict, the second condition must be modified such that, out of an
	abundance of caution, the smallest reasonably likely set of benefits
	must be larger than the greatest reasonably likely set of harms to
	accrue. This is because systemic harm is very difficult to rectify. At
	worst, it looks like the tribal epistemics that are described by David
	Roberts\footnote{\textit{vide}
		https://www.vox.com/policy-and-politics/2017/3/22/14762030/donald-trump-tribal-epistemology},
	where institutions are abandoned wholesale to the point that having
	common institutions at all which help us to find the truth, like
	universities (whose research helps to improve society by telling us
	about problems and how to fix them), neutral government institutions
	(who have a similar function - not to deliberate about what ought to be
	the case but to implement ought-conditions given to them by governments
	and tell the public about what \textit{is} so that they can figure out
	what ought to be) and, just as important if not more so, a free press,
	becomes increasingly unworkable due to a lack of public buy-in. Notably,
	there are few easy solutions once this road is gone down. It is unclear
	how, if at all, the mainstream press is to regain the trust of those who
	are convinced that it is now in the pay of globalist conspirators. All
	neutral institutions are potentially subject to the same level of
	abandonment, which greatly diminishes their utility. To err on the side
	of caution, therefore, is only a logical conclusion of this\footnote{See
		other discussion related to the precautionary principle; the
		precautionary principle is a similar principle, where certain threats,
		although distant and somewhat unquantifiable, can be justified to
		restrict actions, due to their large magnitude. It applies to health
		and the environment, instead of a risk of abandonment of institutions.}.
	
	If the headmaster insists on using his platform for the propagation of
	dialectic material in the manner he has done so, it's clear that he
	risks incurring systemic harm of the same sort.
	
	First, those who agree with his statements may benefit. However, they
	will not particularly benefit - a headmaster's consolation is not
	insubstantial but it is not clear that it is particularly large.
	Moreover they are capable of reading things they agree with in other
	places, at other times, by authors more reputable (since their
	livelihoods may be based on writing about the economics which the
	headmaster of a school is not \textit{ex oficio} concerned with). Hence we
	can largely discount this benefit.
	
	Second, those who disagree with his statements are less likely to
	identify with the headmaster, and, by extension, the school. This is
	problematic in two ways. First, it undermines respect for the
	institutions of the school - they are viewed not as neutral arbiters but
	as part of ``the other side'' - in this case, the ``liberal'' side. This
	is problematic in the same way as the rubbish collection example
	describes. Cooperation with school authorities is important in a number
	of respects - it helps to maintain safety, it qualitatively ameliorates
	learning, it results in smoother administration and reduces costs, and
	so on. The headmaster sacrifices this not only at his peril but at the
	school's peril, for systemic harms like this will last after his
	departure. Second, it reduces the openness with which views are
	expressed. This is a somewhat independent harm. Views which aren't heard
	are less likely to be held because probabilistically they are unlikely
	to be independently be developed; their lack of audience is orthogonal
	to their truth. This violates the second principle established in the
	first section of this essay.
	
	Third, this harm is compounded by an innate adolescent contrarianism,
	and unjustifiably permits those who go against liberal orthodoxy to
	claim the intellectual high ground. That adolescent contrarianism is
	beyond the scope of this essay; assuming that it exists, heavy and
	burdensome imposition appears to be unhelpful in changing attitudes for
	the better. Moreover the headmaster's insistent use of rhetoric, as
	opposed to analytic argument grounded in an instrumental rationality
	whose values can broadly be accepted by those whom he ought to seek to
	convince\footnote{If he is not attempting to convince, he is merely
		virtue-signalling, which is clearly not worth our time.} allows his
	lack of analysis to be ridiculed as indicative of poor reasoning on the
	part of all those who seek to promote, for example, free trade, or
	socially liberal values.
	
	\subsection{Is neutrality possible?}\label{is-neutrality-possible}
	
	In short, not particularly, but we have a workaround.
	
	All statements either contain an implicit or explicit assumption as to
	what is important or admit their unimportance. Consider statements about
	fire alarms. This assumes that fire alarms are impossible (because fire
	alarms help to prevent death which is bad, and that death is bad is
	important).
	
	Implicit assumptions as to what is important do not necessarily make a
	statement unimportant. We take lots of things as implicit;
	mathematicians take some very basic mathematics as implicit, eg. the law
	of identity, but we do not criticise them for that (much).
	
	Given this, ``neutral statements'' also contain dialectic statements
	themselves as to what is important. These are in the same class as
	``ought'' statements - ``$x$ is important'' can broadly be mapped
	to ``one ought to prioritise $x$'', and ``$x$ is more
	important than $y$'' can be mapped to ``one ought to prioritise
	$x$ over $y$''.
	
	Hence by default the principles stated at the beginning of this essay
	for non-established claims apply.
	
	However, there are four exceptions to this rule.
	
	First, if everyone in a community agrees on a specific ``is'' or
	``ought'' statement (eg. ``education is utile'', ``students will take
	GCSE examinations'' and ``water will be provided to students''),
	reminders, clarifications and so on are all permissible, without
	specific reference or resort in line with the principles enumerated
	earlier. This is because the implementation of a pre-agreed ``is'' or
	``ought'' oughtn't to require agreement again.
	
	Second, we take assent without belief to constitute agreement.
	Attendance of this school is voluntary; the vast number of schools which
	exist and the lack of coercion on the part of the school mean that we
	can take the school to be engaged in an ordinary contractual interaction
	with parents and schoolchildren\footnote{Contrast, for example, the
		state, which may not do certain things due to its position as already
		here. Note further the inapplicability of a potential argument against
		claims reliant on choice: in an example where there are ten prisoners
		in a cell and one key, the ten prisoners are not said to be free even
		though they may be each individually able to escape, but the situation
		is disanalogous - there is no necessity that one joins the school, and
		there is sufficient liquidity experienced, at least by almost all
		those who go to this school, to say that this argument would not apply
		anyway because everyone within the school is able to choose somewhere
		else.}. Hence that the ethos of the school involves the protection of
	the wellbeing of all students to some equal degree in outcome in all
	students means, for example, that even if one were to think
	otherwise\footnote{Consider certain homophobic religions.} and disagree
	the school may still act based on it, and so pronouncements in assembly,
	for example, about such matters are also permissible.
	
	Third, what is trivially derived from anything in the former two also
	ought to be included as an exception. This is because parents accept
	that there is a degree of uncertainty as to the outcomes of a school's
	axiology, as it were. They accept the school's axiological premises,
	instead of particular outcomes.
	
	Fourth, in a school environment, certain things are necessarily accepted
	as part of the former three exceptions. Schools accept mainstream
	academic thought in that they teach it. This is accepted by attendance.
	It is therefore permissible to, for example, take as established that
	there exists anthropogenic climate change, even if parents have not
	accepted this, by the second principle\footnote{See footnote 1 for
		further clarification of the idea of a consensus.}.
	
	The former three exceptions can be applied to other institutions too. A
	rubbish collection company may promote respecting recycling sorting
	procedures without falling foul of the principles established before.
	
	\subsection{Conclusion}\label{conclusion}
	
	First, we establish five principles which apply to the strength and
	presentation of dialectic compulsorily attended events. We also
	establish that the headmaster failed to meet four of the five principles
	at various times, and what he did failed to qualify for an opt-out as
	``established''. Second, we show that overtly dialectic compulsorily
	attended events cause three harms, the last of which is systemic. Given
	this systemic nature, two principles apply: first, that all alternatives
	should be maximally used, and second that if this is insufficient to
	avoid using publicly shared institutions, we should use extra caution,
	and so the smallest reasonably likely set of benefits should exceed the
	largest reasonably likely set of harms. Third and finally, we show that
	there are some opt-outs based on what is pre-agreed.
	
	There is a broader problem here, which has not been identified so far.
	It relates not so much to the principles espoused here as the sense of
	entitlement which is felt by those who exploit common institutions in
	this way. It is shared by the proms conductor who felt that the beauty
	of Beethoven's Ninth was insufficient as a plea for our common humanity
	and so his vastly inferior words had to compensate and those who thought
	his words appropriate and utile. Its manifestation is in the growing
	departure from any pretense of neutrality on the part of some of the
	institutions which need it most - schools and universities, the press
	and other common institutions. It is a complete ignoring of systemic
	considerations - a sand castle of an institution is one which is
	brittle, and when cracks start to appear, that a little water will
	appear to fix is no consolation, for the sand castle is doomed thence to
	fall.

\end{multicols}
\end{document}
